\documentclass[hyperref,german,proseminar]{cgvpub}
%weitere Optionen zum Erg�nzen (in eckigen Klammern):
% 
% female	weibliche Titelbezeichnung bei Diplom
% bibnum	numerische Literaturschl�ssel
% final 	f�r Abgabe	
% lof			Abbildungsverzeichis
% lot			Tabellenverzeichnis
% noproblem	keine Aufgabenstellung
% notoc			kein Inhaltsverzeichnis
% twoside		zweiseitig
\author{Roland Raytracer}
\title{Titeltext}
\matno{1234567}
\betreuer{Dr. B. Spline}
\bibfiles{literatur}
\problem{Text der Aufgabenstellung...}
\copyrighterklaerung{Hier soll jeder Autor die von ihm eingeholten
Zustimmungen der Copyright-Besitzer angeben bzw. die in Web Press
Rooms angegebenen generellen Konditionen seiner Text- und
Bild"ubernahmen zitieren.}
\acknowledgments{Die Danksagung...}
\begin{document}

\section{eine Grafik}
\begin{figure}[htbp]
	\centering
		\includegraphics{test.png}
	\caption{beschriftung}
	\label{fig:diplominf}
\end{figure}


\subsection{Etwas Mathe}

\[
\sum_{i=1}^{100}x_i
\]
noch mehr text
\subsubsection{Verweise auf Literatur}
So kann ich Literatur aus literatur.bib zitieren: \cite{kochbuch}. 

\paragraph{etwas quelltext}

l��
\begin{figure}[htbp]
\begin{lstlisting}[frame=trbl]
//comment
for(int i = 0; i < 100;i++)
{
test(i);
}
\end{lstlisting}
\end{figure}

text



\end{document}