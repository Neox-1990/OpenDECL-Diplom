\documentclass[hyperref,german,diplommedieninf]{cgvpub}
%weitere Optionen zum Erg�nzen (in eckigen Klammern):
% 
% female	weibliche Titelbezeichnung bei Diplom
% bibnum	numerische Literaturschl�ssel
% final 	f�r Abgabe	
% lof			Abbildungsverzeichis
% lot			Tabellenverzeichnis
% noproblem	keine Aufgabenstellung
% notoc			kein Inhaltsverzeichnis
% twoside		zweiseitig


\author{Ronald Gro�mann}
\title{Open Display Environment Configuration Language}
\birthday{6. April 1990}
\placeofbirth{Sebnitz}
\matno{3507432}
\betreuer{Dr. rer. nat. Sebastian Grottel}
\bibfiles{literatur}
\problem{
Die Anzahl der Multi-Display-Installationen nimmt in beinahe jeder Umgebung zu: angefangen von Desktopsystemen mit mehreren Monitoren, �ber Projektionsfl�chen mit mehreren Projektoren (aka Powerwalls) bis hin zu komplexen VR-Installationen wie Caves. Oft ben�tigen solche Display-Systeme leistungsstarke, parallel GPU-Cluster zur Bilderzeugung, allein um der notwendigen Pixelf�llrate gerecht werden zu k�nnen. Zus�tzlich kommen in solchen Anlagen �blicherweise auch Tracking-Systeme zum Einsatz, die den Benutzer, also die physische Welt, mit den dargestellten Szenen, der k�nstlichen Welt, verbinden; auch im Desktop-Bereich, z.B. durch Windows Kinect oder Leap Motion. Um solche Systeme zu betreiben bedarf es komplexer Software. Diese ist oft im akademischen Umfeld entwickelt. Solche Software muss umfassend konfiguriert werden, einerseits was die verf�gbare Rechnerinfrastruktur betrifft (welche Computer sind mit welchen Ausgabeger�ten verbunden, welche Computer dienen rein zur entfernten Bilderzeugung und wie sind die Rechner miteinander vernetzt), andererseits auch was die logischen und physikalischen Parameter der Ausgabeger�te betrifft (virtuelle Desktop-Gr��en und Teile einzelner Beamer, physikalische Anordnung von Display oder Projektoren und Abgleich mit den Raumkoordinaten eines Benutzer-Trackings). Allerdings hat sich f�r diese Konfigurationen bisher kein Standard entwickelt.\\

In dieser Arbeit soll ein Vorschlag f�r so ein standardisiertes Konfigurationsformat auf Basis von XML entwickelt werden. Mittels XSLT soll es m�glich sein, aus einer XML-Datei Konfigurationsdateien f�r unterschiedliche Programme zu erzeugen. Ein graphischer interaktiver Editor soll das Erstellen und Bearbeiten der XML-Dateien, sowohl der strukturellen Eigenschaften (Compute-Cluster-Architektur, inklusive GPUs und Display-Anschl�ssen) als auch der 3D physikalischen Eigenschaften (Display-, Projekt-Setup) anschaulich und einfach erm�glichen. Das XML-Format muss ?sauber? durch Namespaces aufgeteilt und erweiterbar sein, was auch durch entsprechende Funktionen im graphischen Editor reflektiert werden muss (z.B. muss es im Editor m�glich sein, eigentlich nicht unterst�tze Tags editieren und beim Abspeichern erhalten zu k�nnen). Die Erstellung von XSLT-Dateien muss durch den Editor NICHT unterst�tzt werden, ihre Anwendung zum Export der Konfiguration in entsprechende andere Formate jedoch schon.\\

Folgende Hardware-Installationen m�ssen unterst�tzt werden:
\begin{enumerate}
\item Desktop-Computer mit mehreren Monitoren (mindestens zwei) die nicht in einer gemeinsamen Ebene stehen.
\item Stereo-Powerwall durch zwei Beamer betrieben an einem Rechner (Powerwall an der Professur CGV)
\item Gro�fl�che Displaywand mit mehreren Panels (Displaywand an der Professur Multimedia-Technologie)
\item F�nf-Wand-CAVE mit zehn Beamer (im VR-Labor des Lehrstuhls Konstruktionstechnik / CAD)
\end{enumerate}

Hierbei m�ssen die physikalischen Display- und Projektionsanordnungen unterst�tzt werden, und zus�tzliche Infrastruktur, wie z.B. Computer, GPUs, Tracking-Systeme etc., sollen so weit wie m�glich unterst�tzt werden.\\

Die Hardwareinstallationen sollen in ihrem physikalischen Raum, in Metern, frei definierbar sein. Ist kein Benutzertracking vorhanden, so muss eine Standardposition f�r den Benutzer (Blickpunkt) konfigurierbar sein.\\

Die Konfiguration der Rechnerinfrastruktur muss mindestens die Computer enthalten, die direkt an die Ausgabeger�te angeschlossen sind. Ihre Netzwerkverbindungen untereinander sollten enthalten sein. Eventuelle Compute-Cluster zur Bilderzeugung und ihre Netzwerkverbindungen untereinander, sowie zu den Ausgaberechnern sollten ebenfalls konfigurierbar sein.\\

Folgende Software muss unterst�tzt werden, indem ihre Konfigurationsdateien, mindestens aber der entsprechend dieser Arbeit relevante Teil der Konfigurationsdateien, erzeugt werden kann:
\begin{description}
\item[a,] MegaMol, bzw. mittels einer kleinen Bibliothek jede an der TUD selbst entwickelte Software
\item[b,] Paraview
\item[c,] Equalizer (optional)
\end{description}

Weitere Software soll nach Absprache mit dem Betreuer ebenfalls unterst�tzt werden.\\

Die Bearbeitung erfolgt mit diesen Teilzielen:
\begin{itemize}
\item Anforderungsanalyse auf Basis der vorgegebenen Display-Hardware und Konfigurationsspezifikationen der einzusetzenden Software
\item Literaturrecherche zur Large-Display- und VR-Software-Middleware und Konfigurationen. Auch zu allgemeinen Arbeiten zur Kalibrieren und Konfiguration solcher Hardware-System
\item Spezifikation der XML-basierten Konfiguration
\item Umsetzung des geforderten Editor-Prototypens inklusive XSLT-basiertem Export der Konfigurationen
\item Evaluierung im Kontext der vorgegebenen Display-Systeme durch Darlegung und Durchf�hrung des kompletten Konfigurationsprozesses anhand der vorgegebene Software
\item Optional: Erweiterung des Editors um weitere Funktionalit�ten zur semi-automatischen Erzeugung der Konfigurationsdateien
\item Optional: Code-Bibliothek zur direkten Nutzung der Konfigurations-Xml-Datei
\item Optional: Untersuchung weiterer Display-Konfigurationen (z.B. gekr�mmter Projektionen) und weiterer Visualisierungs- und VR-Software
\end{itemize}
}
\copyrighterklaerung{Hier soll jeder Autor die von ihm eingeholten
Zustimmungen der Copyright-Besitzer angeben bzw. die in Web Press
Rooms angegebenen generellen Konditionen seiner Text- und
Bild"ubernahmen zitieren.}
\acknowledgments{Die Danksagung...}

\abstracten{abstract text english}
\abstractde{ Zusammenfassung Text Deutsch}

\begin{document}
%Kapitel 1 #######################
\chapter{Problemstellung}
Grunds�tzliche Herausforderungen bei Tiled/Multi Display Umgebungen\\

\section{Problemfelder}
Aufteilung der Problemfelder\\

%Kapitel 2 #######################
\chapter{Verwandte Arbeiten}

\section{Problemfeld 1}

\section{Problemfeld 2}

\section{Problemfeld 3}

%Kapitel 3 #######################
\chapter{Anforderungsanalyse}
was habe ich analysiert\\
auf welche Aspekte habe ich mich konzentriert\\
was habe ich weniger oder gar nicht analysiert\\

\section{Hardwareanalyse}
mein Desktop System\\
CGV Stereowall\\
MT Powerwall\\
CAVE\\

\section{Softwareanalyse}
Megamol\\
Paraview\\
Equalizer\\

%Kapitel 4 #######################
\chapter{Konzeption}
Auf was konzentriert sich mein Ansatz, wo liegt der Fokus\\
Was soll Abgebildet werden\\
Was wird Beschrieben

%Kapitel 5 #######################
\chapter{OpenDECL Spezifikation}
Spezifikation liegt als XSD vor, siehe Anhang\\

\section{Ansatz}
Wie habe ich geplant die Konzeption umzusetzen\\
Wo habe ich mich orientiert, bzw. Habe ich andere Vorlagen gehabt (stichwort equalizer, VRUI,...)

\section{Aufbau}
Es folgt die konkrete Umsetzung der Spezifikation\\
Ich gehe Schritt f�r Schritt die Elemente durch, wobei das Wurzelelement (openDECL) drei verschiedene Untelemente besitzen kann, die hier als Anhaltspunkt f�r die weitergehende Beschreibung genutzt werden.\\

\subsection{node}
Was soll * darstellen\\
Welche Atrribute beschreiben was + optional vs required\\

\subsubsection{graphics-device}
Was soll * darstellen\\
Welche Atrribute beschreiben was + optional vs required\\

\subsubsection*{port}
Was soll * darstellen\\
Welche Atrribute beschreiben was + optional vs required\\

\subsubsection{network-device}
Was soll * darstellen\\
Welche Atrribute beschreiben was + optional vs required\\

\subsection{network}
Was soll * darstellen\\
Welche Atrribute beschreiben was + optional vs required\\

\subsection{display-setup}
Was soll * darstellen\\
Welche Atrribute beschreiben was + optional vs required\\

\subsubsection*{vector}
Was soll * darstellen\\
Welche Atrribute beschreiben was + optional vs required\\

\subsubsection{user}
Was soll * darstellen\\
Welche Atrribute beschreiben was + optional vs required\\

\subsubsection{display}
Was soll * darstellen\\
Welche Atrribute beschreiben was + optional vs required\\

\subsubsection*{physical}
Was soll * darstellen\\
Welche Atrribute beschreiben was + optional vs required\\

\subsubsection*{virtual}
Was soll * darstellen\\
Welche Atrribute beschreiben was + optional vs required\\

%Kapitel 6 #######################
\chapter{XSLT Konfigurationsgenerierung}
was ist zu beachten\\
wie viel logik kann man in XSLT reinstecken, wo liegen die Grenzen?\\

\section{Beispiel}
einfaches Beispiel CGV Stereowall mit Megamol\\
komplexeres Beispiel MT Powerwall mit Paraview\\
hypothetisches Beispiel f�r CAVE mit Paraview\\

%Kapitel 7 #######################
\chapter{Editor}

\section{Funktionen}
Was kann Editor\\
Wo liegen Grenzen des Editors\\
kleines Manual\\

\section{Umsetzung}
Grober Aufbau\\
Datenhaltung im Hintergrund\\
Zugriff via Interface\\

%Kapitel 8 #######################
\chapter{Evaluation}

%Kapitel 9 #######################
\chapter{Diskussion}

%Kapitel 10 ######################
\chapter{Ausblick}

\end{document}